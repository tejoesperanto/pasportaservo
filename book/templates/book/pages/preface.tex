\invisiblesection{Prefaco}
\thispagestyle{plain}
{\parskip=1em

Karaj gastoj kaj gastigantoj!

\vspace{.8em}

En la Jaro de Zamenhof, 2017 Pasporta Servo reviviĝas kaj revenas en niajn vivojn. Vi denove povas teni la libron en viaj manoj!

Pasporta Servo praktike signifas, ke tiuj homoj kiuj gastigas akceptas homojn en sia hejmo por almenaŭ unu nokto senpage. Sed la komunumo de Pasporta Servo estas multe pli ol nur la sumo de tiuj pasigitaj noktoj tutmonde. Estas multege da Esperanto-parolantoj, kiuj lernis la lingvon pro Pasporta Servo. Ili unue uzis ĝin por vojaĝi, kaj nun jam mem kiel gastigantoj enestas. Vi trovos gastigantojn, kiuj jam enestis 30 jarojn antaŭe ankaŭ kaj daŭre entuziasme gastigas.

Ni scias, ke ne ekzistas Esperanto-lando, do ni mem devas krei la oportunojn por renkontiĝi, uzi nian lingvon. Vojaĝado, ekkonado de landoj tra la okuloj de lokanoj, esti akceptata en la hejmo kiel malnova amiko estas multe pli ol simple listo de adresoj sugestas.

Pasporta Servo, la libreto revenas post iom longa silento. Tamen, intertempe multe da aktivuloj strebis restarigi la sistemon. Uzis sociajn retejojn aŭ aliajn gastigoservojn por gardi la spiriton de gastigado de Pasporta Servo. Dankon al vi!

La libreto revenas, kaj estas multege da laboro malantaŭ ni. Multaj demandas min: kial gravas la libro nun, en 2017? Kiel ne sufiĉas la retejo aŭ apo por poŝtelefono? La respondo estas facila: ĉi tiu libro enhavas nur plene konfirmitajn adresojn, nur homojn kiuj daŭre deziras resti gastigantoj. Kiuj estas en la libro, ĉiu persone respondis al niaj leteroj aŭ alvokoj.

En la retejo estas multe pli da gastigantoj, kaj ni provis kontroli kiom eble plej multe, sed tiuj, kiuj aperas ĉi tie rapide respondas, emas gastigi, kaj elkore bonvenigas vin. Vi volas tiajn gastigantojn, ĉu ne?

Tamen ankaŭ la retejo bone funkcias. Jam plurmil personaj mesaĝoj estis senditaj ekde la relanĉo en novembro 2014. Do, nun venas la tempo danki la kompilanton de la libro, kaj ĉefprogramiston de la projekto: \name{Baptiste}{Darthenay}, kiu persone faris pli ol tri kvaronojn de la tuta projekto, kaj pro kiu vi nun povas teni la libron en viaj manoj.

Mi tre dankas la laboron de la Landaj Organizantoj kaj helpantoj, kiuj kontaktis la gastigantojn. Dankon al TEJO, ke kunlabore kun UEA dungis Baptiste kaj min por 4 monatoj, tiel ni povis trankvile labori interalie pri la libreto.

Fine, dankon al la gastigantoj kaj vi, ke vi aĉetis la libron! Sciu, ke tiel vi rekte subtenas Pasportan Servon.

Ni daŭre laboros por evolui la libron, la retejon kaj la komunumon de Pasporta Servo. Ni esperas, ke la libro eĉ pli vigligos la Esperanto-komunumon kaj kontribuos al neforgeseblaj vojaĝoj, renkontiĝoj, kaj babiladoj!

Bonvenon al Pasporta Servo 2017!

\vspace{.8em}

~\hfill\textit{\semibold Stela}
}
