\thispagestyle{plain}
{
  \titleformat{\section}[block]{}{}{0pt}{\huge\textbf}
  \parskip=0.6em

\section{Enkonduko}

\subsection{Kio estas Pasporta Servo?}


Pasporta Servo estas plene volontula internacia gastiga servo per Esperanto. Ĝi estas la ilo, kiu enhavas la gastigantojn, kiuj pretas oferti senpagan tranokteblon en siaj hejmoj por almenaŭ unu nokto. Praktike ĝi estas listo de adresoj ordigitaj laŭ landoj laŭ la Esperanta alfabeto.

Pasporta Servo permesas multe malpli koste vojaĝi tra la mondo. Kie kaj laŭ kiuj kondiĉoj vi tranoktas dependas de la gastiganto. Kelkfoje vi havos nur lokon sur planko, aŭ matracon, sed povas okazi, ke eĉ liton aŭ apartan ĉambron!

Vi aldone povas ricevi helpon de la lokanoj rilate al la restado en la lando: pri transporto, kiujn lokojn viziti, kion manĝi, provi, sperti.

\subsection{La komencoj de Pasporta Servo}

La unua ideo pri la tiel nomata \textit{Programo Pasporto} estiĝis en 1966, de \name{Ruben}{Feldman-Gonzalez} el Argentino. Pasporta Servo laŭ la nuna sistemo aperis unuafoje en 1974, kun 39 gastigantoj, sub gvido de \name{Jeanne-Marie}{Cash} el Francio. Ambaŭ pioniroj daŭre estas gastigantoj de Pasporta Servo.

Pasporta Servo estas eldonaĵo de la Tutmonda Esperantista Junulara Organizo (TEJO).

\subsection{Kial estas utila Pasporta Servo por la gastiganto?}

Por gasto havi tranokteblon senpage estas bonege. Havi gastiganton kun kiu oni povas facile interkompreniĝi parolante Esperanton eĉ pli bone. Tamen Pasporta Servo helpas ankaŭ la gastigantojn. Kelkaj gastigantoj ofte ne havas mem la eblon vojaĝi, praktiki la lingvon aŭ partopreni Esperanto-renkontiĝojn. Do, Pasporta Servo donas la eblecon interligi Esperanto-parolantojn. Tiel vi povas lerni pri aliaj kulturoj ne nepre devante vojaĝi. Konsideru iĝi gastiganto!

\subsection{Kiel vi povas membriĝi en Pasporta Servo?}

Unue, se vi estas gastiganto, vi aŭtomate estas membro de Pasporta Servo.
Se vi deziras gasti, vi jam tenas tiun libron en via mano. Certiĝu, ke vi skribas vian nomon sur la kovrilon!
Nia retejo: www.pasportaservo.org/registrado/ estas tre facile uzebla, vi povas kaj registri vian hejmon, kaj serĉi gastigantojn. Vi iĝas membro plenigante viajn datumojn.
Membreco en Pasporta Servo ne estas ligita al iu ajn alia membreco.
La servo mem estas senpaga, sed ne senkosta. Aĉetado de la libro subtenas la projekton, kaj ajna monsumo helpas pagi la laboron de la programistoj, redaktoroj, presadon de la libro kaj la servilon.

Tra UEA vi povas mendi pliajn PS-librojn aŭ subteni nian laboron per ajna monsumo.
Por vidi pri la pagado, bv. iru al:\\
http://www.uea.org/alighoj/pagmanieroj

Dankon pro via subteno!


\subsection{Kiel vi povas fariĝi gastiganto?}

Vi fariĝas gastiganto aŭ membriĝante ĉe pasportaservo.org aŭ sendante viajn informojn al Centra Oficejo de TEJO. Via aliĝilo devas alveni ĉe la kompilanto antaŭ la limdato indikita sur la aliĝilo por esti certa pri apero en la sekva eldono de Pasporta Servo.

Se vi havas ajnan problemon rilate al la registriĝo, nepre skribu al saluton@pasportaservo.org, ni volonte helpas al vi!

\subsection{Bazaj kondiĉoj}

La plej grava kondiĉo estas, ke ĉiu membro de Pasporta Servo (gastoj kaj gastigantoj) devas paroli Esperanton. Tio estas la bazo de la Servo, kaj ĝi estas nenegocebla. Povas okazi, ke vi ne parolas tre bone, sed vi devas povi komuniki en Esperanto.

Tiu kondiĉo ne estas nepra por kunloĝantoj kaj kunvojaĝantoj, krom se gastiganto mencias en siaj kondiĉoj, ke akceptas nur Esperanto-parolantajn gastojn.

La gastiganto devas oferti minimume unu nokton senpage por ĉiu gasto. Povas eventuale peti monon kontraŭ matenmanĝo aŭ aliaj servoj, sed tio devas esti klare skribite en la kondiĉoj aŭ strikte nur laŭ interkonsento antaŭ la alveno.

Ĉiu kondiĉo de la gastiganto devas esti respektita. Memoru, temas pri fido kaj bonvolo. Se vi ion ne komprenas, aŭ pri io ne certas, prefere demandu la gastiganton por klarigo.

La gasto neniam rajtas alveni sen antaŭa kontaktado. Se tio okazas, ne surprizu, se vi, la gasto estos forsendita.


\subsection{Ĝis kiam validas la libro?}

Ĉiu adreslibro validas ĝis la sekva eldono. Ĝi devas aperi ĉiu jare, la adresoj validas en la jaro, kiu troviĝas sur la kovrilo.

\subsection{Konfirmado de la adresoj}

En tiu ĉi libro troviĝas nur gastigantoj, kiuj persone konfirmis sian aperon en la eldono. Ĉiu jare ni sendas rememorigilon por konfirmigi, ke la adreso de la gastiganto daŭre estas aktuala kaj ankaŭ fakte la emo gastigi.

Kompare al la reta versio de Pasporta Servo, ĉi tie enestas nur adresoj, kiuj estas konfirmitaj kaj kontrolitaj de la Landaj Organizantoj kaj la kompilanto.


\subsection*{Kiel NE uzi la adreslibron?}

Gastigantoj volonte gastigas vin laŭ la menciitaj kondiĉoj, do bonvolu ne misuzi la adreslibron. Pasporta Servo estas fidosistemo. Ĝi funkcias, se la gastigantoj povas fidi vin, ke vi respektas la kondiĉojn, la petojn kaj la regulojn.

Ne uzu la libreton por io ajn krom serĉado de tranoktebloj. Ĝi ne estas por korespondado, varbado ktp. Se vi ne respektas la regulojn bedaŭrinde vi ne rajtos membri en Pasporta Servo.

Se vi deziras trovi korespondantojn vi povas turni vin al

- Esperanto Koresponda Servo: \url{http://www.esperantofre.com/eks/}

- Edukado.net ankaŭ havas dediĉitan paĝon al korespondantoj ĉe: \url{http://edukado.net/komunumo/korespondaservo}

\subsection*{NE por invitiloj!}

La gastigantoj ne povas aranĝi oficialan invitilon. Kiel individuoj, ili nek povas, nek devas doni la necesajn garantiojn. Se vi bezonas invitilon por eniri landon vi kutime bezonas kontakti oficialan institucion de via lando, kutime la Ministerion de Eksterlandaj Aferoj. Gastigantoj nek peras laborlokojn en sia lando, ili ofertas tranokton.
}
