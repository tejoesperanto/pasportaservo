\section{Iom pli pri Pasporta Servo}
\subsubsection{Kio estas Pasporta Servo?}
Pasporta Servo estas servo de la Esperanto-movado por Esperanto-parolantoj.
Ĝiaj membroj havas la eblon ricevi senpagan gastadon. Nun, Pasporta Servo
(www.pasportaservo.org) estas ankaŭ retpaĝaro, uzebla same kiel tiu ĉi libreto.
Tial, ne plu estas absolute nepre havi la libron por esti gasto per Pasporta Servo.
Membreco en Pasporta Servo ne estas ligita al iu ajn alia membreco.
Gastigado laŭ Pasporta Servo ne nepre estas luksa. Ĉiuj variantoj ekzistas, de
surplanka dormosako ĝis la rajto je utiligo de tuta domo. La adreslibro de Pasporta
Servo estas eldonaĵo de Tutmonda Esperantista Junulara Organizo (TEJO).
La unua ideo pri la tiel nomata ‘Programo Pasporto’ estiĝis en 1966, de Ruben
FELDMAN-GONZALEZ el Argentino. Pasporta Servo laŭ la nuna sistemo aperis
unuafoje en 1974, kun 39 gastigantoj, sub gvido de Jeanne-Marie CASH el Francio.
Ambaŭ pioniroj daŭre estas gastigantoj de Pasporta Servo.
\subsubsection{Kiel fariĝi membro?}
Estas du ebloj: aŭ vi membriĝas ĉe www.pasportaservo.org, aŭ vi aĉetas la adreslibron
de la gastigantoj, kaj skribas vian nomon sur la kovrilon. Se vi estas uzanto de la
retpaĝo, nepre menciu ke vi gastigas, kaj ke vi volas aperi en la adreslibro. Nova
adreslibro estas regule eldonata, kaj ĉiu adreslibro validas ĝis la sekva eldono.
Se vi ne scias pri proksima libroservo aŭ klubo kiu povas vendi la adreslibron al vi,
rigardu sub Centra Distribuanto. Gastigantoj aŭtomate fariĝas membroj.
\subsubsection{Kiel fariĝi gastiganto?}
Vi fariĝas gastiganto aŭ membriĝante ĉe la retpaĝaro www.pasportaservo.org aŭ
sendante informojn al Centra Oficejo de TEJO. Aliĝiloj devas alveni ĉe la kompilanto
(vidu poste) antaŭ la limdato indikita sur la aliĝilo por esti certa pri apero en la tuj
sekva eldono de Pasporta Servo. Sendante la aliĝilon pli malfrue, vi riskas aperi nur
en la eldono post la tuj sekva. Aliĝilo troviĝas en la adreslibro, kaj haveblas de la
Landa Organizanto aŭ la Centra Distribuanto. Ne estas limdato por membriĝi rete.
\subsubsection{Kondiĉoj}
Ĉiuj membroj de Pasporta Servo (gastoj kaj gastigantoj) devas scipovi Esperanton.
Tiu kondiĉo ne estas nepra por kunloĝantoj kaj kunvojaĝantoj.
La adreslibro estas persona kaj validas nur por vi mem. Vi ne rajtas pruntedoni
ĝin al alia persono. Por ke la adreslibro validu, ĝi devas havi vian nomon (tiu de la
posedanto) sur la frontpaĝo. Montri la adreslibron al via gastiganto je unua renkonto
ne plu estas nepre.
2011	
\subsubsection{Konfirmo}
Ĉiuj gastigantoj devas konfirmi sian mencion por la venonta eldono. Tio necesas por
eviti malaktualajn adresojn, kondiĉojn aŭ informojn. Vi aŭtomate ricevos memorigilon
kun respondilo ĉirkaŭ du monatojn antaŭ la limdato de la venonta eldono. Simple
resendu la respondilon al la kompilanto, kun eventualaj ŝanĝoj aŭ korektoj. Se vi
ne ricevas memorigilon (ekz-e pro translokiĝo) vi povas konfirmi per aliĝilo. Atentu!
Tiuj kiuj ne konfirmas antaŭ la limdato aŭtomate elfalos!

\subsection{Kiel NE uzi la adreslibron?}
Gastigantoj volonte gastigas vin laŭ la menciitaj kondiĉoj, sed bonvolu ne misuzi
la adreslibron. Jam tro da gastigantoj malaliĝis pro la ricevado de leteroj kiuj petas
ĉiajn aferojn krom gastigado. Ne uzu la adreslibron por korespondado, vendado,
petado de mono, serĉo de edz(in)o aŭ entreprenaj proponoj. Bonvolu ne malhelpi al
Pasporta Servo! Eble la Delegita Reto de UEA povas pli bone helpi al vi.
\subsubsection{NE por korespondado!}
Pasporta Servo ne estas adreslibro de korespondpetantoj. Multaj gastigantoj
ĉagreniĝas kiam oni traktas ilian aperon en Pasporta Servo kiel peton pri kores­
pon­dado. Por korespondpetoj vi povas turni vin al Koresponda Servo Mondskala
(vidu Utilaj adresoj, p. 10).
\subsubsection{NE por invitiloj!}
Krome, gastigantoj ne povas aranĝi oficialan invitilon. Kiel individuoj, ili ne povas
doni la necesajn garantiojn. Por invitilo kontaktu oficialan instancon. Gastigantoj
ankaŭ ne peras laborlokojn en sia lando.
